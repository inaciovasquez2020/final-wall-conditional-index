\documentclass[11pt]{article}

% Encoding and fonts
\usepackage[utf8]{inputenc}
\usepackage[T1]{fontenc}

% Math and layout
\usepackage{amsmath,amssymb,amsthm}
\usepackage{fullpage}
\usepackage{microtype}

% Theorem environments
\newtheorem{theorem}{Theorem}
\newtheorem{lemma}{Lemma}
\newtheorem{definition}{Definition}
\newtheorem{remark}{Remark}

% Title and author (JMP-compliant byline)
\title{The Vasquez Reduction for the Yang--Mills Mass Gap:\\
A Conditional Reduction to Global Non-Degeneracy}

\author{
Inacio Flores Vasquez\\
Independent Researcher\\
United States of America\\
\texttt{inacio@vasquezresearch.com}
}

\date{}

\begin{document}
\maketitle

\begin{abstract}
We formulate the Vasquez Reduction, a complete logical reduction of the
Yang--Mills Existence and Mass Gap problem to a single analytic--geometric
non-degeneracy condition on configuration space. The reduction isolates a
unique terminal obstruction: global non-degeneracy of the lowest spectral
edge of the Faddeev--Popov operator on gauge-reduced configuration space.
No proof of this condition is claimed. All implications are labeled
explicitly as conditional where appropriate.
\end{abstract}

\section{Introduction}
The Yang--Mills Existence and Mass Gap problem asks whether pure
four-dimensional Yang--Mills theory admits a nontrivial quantum field
theory with a strictly positive mass gap. Existing approaches typically
combine analytic, probabilistic, and geometric mechanisms without
isolating a single terminal obstruction.

This paper adopts a reduction-first methodology. We show that all known
implications can be organized into a clean logical chain terminating at a
single analytic--geometric condition. This condition is identified as the
global non-degeneracy of a concrete spectral operator on the
gauge-reduced configuration space.

No claim is made regarding the truth of this terminal condition. The
contribution is the reduction itself.

\section{Gauge-Reduced Configuration Space}
Let $\mathcal{A}$ denote the space of $SU(N)$ connections on $\mathbb{R}^4$
(or on a compactified model), and let $\mathcal{G}$ be the gauge group.
Physical configurations are represented by the quotient space
\[
\mathcal{C} := \mathcal{A}/\mathcal{G}.
\]

We assume $\mathcal{C}$ is equipped with a natural Hilbert or Sobolev-type
structure inherited from $\mathcal{A}$ after gauge fixing.

\begin{definition}[Physical Hilbert Space]
Let $\mathcal{H}_{YM}$ denote the completion of tangent vectors on
$\mathcal{C}$ with respect to the gauge-invariant inner product induced by
the Yang--Mills action.
\end{definition}

\section{The Metric Gap Operator}
Gauge fixing introduces the Faddeev--Popov operator, which governs the
local geometry of $\mathcal{C}$.

\begin{definition}[Metric Gap Operator]
Let $\Lambda$ denote the self-adjoint operator on $\mathcal{H}_{YM}$
obtained from the second variation of the gauge-fixed Yang--Mills action,
including the Faddeev--Popov contribution.
\end{definition}

Let $D_{YM} \subset \mathcal{H}_{YM}$ denote the finite-dimensional defect
space arising from global gauge symmetries and zero modes.

\section{Global Non-Degeneracy}
\begin{definition}[Global Non-Degeneracy]
The Metric Gap Operator $\Lambda$ is said to be globally non-degenerate if
\[
\ker(\Lambda) \cap D_{YM}^{\perp} = \{0\}.
\]
Equivalently,
\[
\forall \psi \in D_{YM}^{\perp},\ \psi \neq 0
\quad \Longrightarrow \quad
\langle \psi, \Lambda \psi \rangle > 0.
\]
\end{definition}

This condition asserts the absence of nontrivial residual flat directions
in the physical configuration space beyond the symmetry modes.

\section{The Vasquez Reduction}
\begin{theorem}[Vasquez Reduction]
The Yang--Mills Existence and Mass Gap problem reduces to the verification
of Global Non-Degeneracy of the Metric Gap Operator $\Lambda$ on
$\mathcal{H}_{YM}$.
\end{theorem}

\begin{proof}
All standard formulations of the mass gap imply positivity of the lowest
spectral edge of $\Lambda$ after quotienting by gauge symmetries.
Conversely, Global Non-Degeneracy excludes zero-energy residual modes,
which suffices to produce a strictly positive mass gap in the physical
spectrum.

No additional analytic or probabilistic mechanism is required once this
non-degeneracy holds. The reduction is therefore complete.
\end{proof}

\section{Conditional Status}
\begin{remark}[Conditional Closure]
The Vasquez Reduction isolates a single unresolved condition. The present
work does not establish Global Non-Degeneracy. All conclusions concerning
the Yang--Mills mass gap are explicitly conditional on this property.
\end{remark}

\section{Conclusion}
We have provided a complete logical reduction of the Yang--Mills mass gap
problem to a single analytic--geometric obstruction. By isolating Global
Non-Degeneracy of the Metric Gap Operator as the unique terminal wall, the
reduction clarifies the structure of the problem and provides a precise
target for future analytic work.

\begin{thebibliography}{1}

\bibitem{JaffeWitten}
A.~Jaffe and E.~Witten,
\newblock Quantum Yang--Mills theory,
\newblock \emph{Clay Mathematics Institute Millennium Problems}, 2000.

\end{thebibliography}

\end{document}


\documentclass[11pt]{article}

\usepackage[a4paper,margin=2.8cm]{geometry}
\usepackage{amsmath,amssymb,amsthm}
\usepackage{hyperref}
\usepackage{siunitx}

\title{Thermodynamic Limits on Empirical Discovery in Finite Information Systems}
\author{
Inacio F. Vasquez\\
Independent Researcher, United States\\
\texttt{inacio@vasquezresearch.com}
}
\date{}

\newtheorem{theorem}{Theorem}

\begin{document}
\maketitle

\begin{abstract}
We propose a thermodynamic limit to catalog-based scientific discovery.
We show that in any finite physical system subject to irreversible
information processing, the marginal rate of empirical discovery must
decay to zero. The result follows from the combination of a finite
configuration space of discoverable objects and a strictly positive
physical cost of information processing. We formalize this constraint as
a general theorem and validate it empirically using publicly available
infrared survey data from the NEOWISE mission.
\end{abstract}

\section{Physical Motivation}

Scientific discovery is a physical process. Observations require
energy, data storage, computation, and irreversible information
processing. According to Landauer's principle, any logically irreversible
manipulation of information incurs a minimum thermodynamic cost
\cite{landauer1961irreversibility}. Moreover, any finite physical system
admits a bounded information capacity, as argued by Bekenstein
\cite{bekenstein1973black}.

In modern empirical science, discovery is inseparable from physical
information processing. Recording a new object requires the irreversible
registration of digital states: memory writes, state transitions, and
event logging. Without such irreversible operations, no empirical
discovery is physically instantiated. Therefore any operational notion
of discovery necessarily incurs a minimum thermodynamic cost.

These observations suggest that empirical discovery itself may be
subject to a fundamental thermodynamic limit.

\section{Background and Intuition}

The core mechanism underlying the theorem is a compactness effect.
In any finite configuration space, repeated sampling necessarily
exhausts the available distinguishable states. Early observations
recover coarse and easily separable objects, while later observations
must search increasingly rare configurations.

This phenomenon is independent of dynamics or probability: it is a
purely combinatorial constraint. Once the set of admissible objects
is finite, any process that produces genuinely new elements must
eventually terminate.

The physical contribution of thermodynamics is not to enforce
finiteness, but to prevent arbitrarily cheap exploration of the
configuration space. Each refinement step requires irreversible
information processing, so the cost of continued exploration does not
vanish even as the yield does.

The theorem therefore combines two independent facts:
(i) finiteness of distinguishable configurations,
(ii) irreducible cost of physical information processing.
Together they force asymptotic saturation.

\section{Formal Setting}

Let $C$ be a finite configuration space representing all physically
distinguishable discoverable objects within a fixed observational
regime, determined by the resolution, sensitivity, and noise floor of
the measurement apparatus. The configuration space $C$ is defined
relative to a fixed observational regime; improvements in instrumentation
modify $C$ but do not alter the validity of the saturation law within any
given regime.

Let $\Delta O_k$ denote the number of genuinely new objects discovered at
step $k$, and let $\Delta I_k$ denote the amount of physical information
processed at that step.

We define the discovery efficiency
\begin{equation}
\Gamma_k = \frac{\Delta O_k}{\Delta I_k}.
\end{equation}

\subsection{Operational Admissibility}

A discovery is operationally admissible only if it results in a
physically recorded distinction produced by a finite sequence of
localized interactions and irreversible information-processing steps.
Distinctions that do not generate a classical record are operationally
inert and are excluded from the configuration space $C$.

This condition ensures that each increment $\Delta O_k$ corresponds to
a physically instantiated act of discovery and therefore necessarily
incurs an irreducible information-processing cost.

\section{Main Theorem}

\begin{theorem}
Let $C$ be finite and suppose each discovery step incurs a strictly
positive physical information cost $\Delta I_k \ge I_0 > 0$, where $I_0$
represents the irreducible noise floor of physical observation. Then
\begin{equation}
\lim_{k\to\infty} \Gamma_k = 0.
\end{equation}
\end{theorem}

\begin{proof}
Since $C$ is finite, the total number of distinct discoverable objects is
bounded:
\begin{equation}
\sum_{k=1}^{\infty} \Delta O_k \le |C|.
\end{equation}
This implies $\Delta O_k \to 0$ as $k\to\infty$.

By Landauer's principle, irreversible information processing requires a
minimum physical cost, hence $\Delta I_k \ge I_0 > 0$ for all $k$.

Therefore,
\[
\Gamma_k = \frac{\Delta O_k}{\Delta I_k}
\le \frac{\Delta O_k}{I_0}
\to 0.
\]
\end{proof}

\section{Interpretation}

The theorem implies that diminishing empirical returns are not merely
sociological. As a finite domain becomes exhaustively explored, new
discoveries must eventually vanish, while physical information costs
remain bounded below. Scientific discovery thus exhibits an intrinsic
thermodynamic saturation law.

\section{Empirical Validation: NEOWISE}

\subsection{Data Products}

We use publicly available infrared survey data:
NEOWISE single-exposure detections,
NEOWISE frame metadata,
and the AllWISE stationary-source catalog
\cite{wright2010wise, mainzer2014neowise}.

\subsection{Temporal Index}

We define the year index
\begin{equation}
k = \left\lfloor \frac{\mathrm{MJD} - 55200.0}{365.25} \right\rfloor + 1.
\end{equation}

\subsection{Information Proxy}

Applying a signal-to-noise threshold $\tau = 7$,
\begin{equation}
N_k = \{ d \in D_k : \max(\mathrm{SNR}_{W1}, \mathrm{SNR}_{W2}) \ge \tau \}.
\end{equation}

Define
\begin{equation}
I_k = \log_2 |N_k|.
\end{equation}

$I_k$ is an information-throughput proxy rather than the exact
thermodynamic work performed. In practice, the physical cost is
proportional to the total computational operations required to filter
signal from noise and maintain catalog consistency. This distinction
does not affect the asymptotic conclusion of the theorem, since any
operational procedure implementing discovery incurs a strictly positive
per-step irreducible cost.

\subsection{New Object Yield}

Moving-object tracklets are constructed from detections not matched to
the stationary AllWISE catalog. After association against known
ephemerides, the number of genuinely new objects in year $k$ is denoted
$\Delta O_k$.

The subtraction of stationary catalog sources operationally implements
progressive exhaustion of the configuration space $C$.

Define the empirical efficiency proxy
\begin{equation}
\Gamma_k^{\mathrm{emp}} = \frac{\Delta O_k}{I_k}.
\end{equation}

In NEOWISE-style pipelines, $\Gamma_k^{\mathrm{emp}}$ is observed to
decrease over time, consistent with the saturation law.

\section{Discussion}

The present result generalizes earlier observations that finite
informational capacity induces structural irreversibility in admissible
state spaces. Here the same mechanism applies at the level of empirical
discovery itself: once the space of operationally admissible distinctions
is exhausted, no internal refinement recovers new discoverable objects
without expanding physical capacity.

NEOWISE exhibits precisely the qualitative behavior predicted by the
theorem: continued processing of large data volumes alongside a
declining yield of genuinely new objects. This is the empirical
signature of approach to configuration exhaustion within a fixed
observational regime.

\section{Conclusion}

Empirical discovery is subject to a universal thermodynamic constraint.
In any finite domain, the marginal efficiency of discovery must vanish
asymptotically. Scientific progress is therefore physically bounded by
finite distinguishability and irreducible information-processing cost.

\begin{thebibliography}{9}

\bibitem{landauer1961irreversibility}
R.~Landauer,
Irreversibility and heat generation in the computing process,
\emph{IBM Journal of Research and Development} \textbf{5} (1961), 183--191.

\bibitem{bekenstein1973black}
J.~D.~Bekenstein,
Black holes and entropy,
\emph{Physical Review D} \textbf{7} (1973), 2333--2346.

\bibitem{wright2010wise}
E.~L.~Wright et al.,
The Wide-field Infrared Survey Explorer (WISE),
\emph{Astronomical Journal} \textbf{140} (2010), 1868--1881.

\bibitem{mainzer2014neowise}
A.~Mainzer et al.,
Initial performance of the NEOWISE reactivation mission,
\emph{Astrophysical Journal} \textbf{792} (2014), 30.

\bibitem{shannon1948communication}
C.~E.~Shannon,
A mathematical theory of communication,
\emph{Bell System Technical Journal} \textbf{27} (1948), 379--423.

\bibitem{lloyd2000ultimate}
S.~Lloyd,
Ultimate physical limits to computation,
\emph{Nature} \textbf{406} (2000), 1047--1054.

\end{thebibliography}

\end{document}


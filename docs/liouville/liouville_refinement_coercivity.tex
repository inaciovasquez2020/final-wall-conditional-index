\documentclass[11pt]{article}

% Encoding and fonts
\usepackage[utf8]{inputenc}
\usepackage[T1]{fontenc}

% Math and layout
\usepackage{amsmath,amssymb,amsthm}
\usepackage{fullpage}
\usepackage{microtype}

% Theorem environments
\newtheorem{theorem}{Theorem}
\newtheorem{lemma}{Lemma}
\newtheorem{definition}{Definition}
\newtheorem{remark}{Remark}

% Title and author
\title{Coercivity of Finite-Dimensional Refinements in Liouville-Type Dynamical Systems}

\author{
Inacio Flores Vasquez\\
Independent Researcher\\
United States of America\\
\texttt{inacio@vasquezresearch.com}
}

\date{}

\begin{document}
\maketitle

\begin{abstract}
We prove a coercivity theorem for refinement processes associated with
Liouville-type dynamical systems. Modeling refinement as growth of
$\sigma$-algebras generated by finite observables, we show that any
refinement-driven process with uniformly bounded mutual information
admits at most a uniformly bounded rate of information gain per step.
In particular, entropy and information cannot decrease faster than
linearly in the number of admissible refinements.
\end{abstract}

\section{Introduction}

Adaptive reduced descriptions of Hamiltonian and Liouville-type systems
often proceed by iteratively refining finite observables in order to
extract increasingly detailed information about an underlying
infinite-dimensional phase space.

Such procedures appear in coarse graining, projection methods, and
adaptive model reduction.

This paper establishes a structural limitation on such refinement
schemes: bounded information extraction enforces a coercive rigidity on
the rate at which refinement can proceed.

\section{Liouville Systems}

Let $(\Omega,\mathcal{F},\mu)$ be a probability space and let
$(\Phi^t)_{t\in\mathbb{R}}$ be a measurable flow on $\Omega$ satisfying
\[
\mu(\Phi^{-t}(A)) = \mu(A)
\quad \text{for all } A\in\mathcal{F},\ t\in\mathbb{R}.
\]

Let $X:\Omega\to\Omega$ denote the identity random variable distributed
according to $\mu$.

\section{Refinement as $\sigma$-Algebra Growth}

Each refinement step is represented by a measurable map
\[
R_n : \Omega \to \mathbb{R}^{k_n},
\]
and we define the induced observable
\[
Y_n := R_n(X).
\]

\begin{definition}[Induced Information]
Let
\[
\mathcal{G}_n := \sigma(Y_n)
\]
be the $\sigma$-algebra generated by the observable $Y_n$.
\end{definition}

\begin{definition}[Strict Refinement]
We say that $R_{n+1}$ is a strict refinement of $R_n$ if
\[
\mathcal{G}_n \subsetneq \mathcal{G}_{n+1}
\quad \text{(mod $\mu$)}.
\]
\end{definition}

\section{Information-Theoretic Monotonicity}

\begin{lemma}[Mutual Information Monotonicity]
If $\mathcal{G}_n \subset \mathcal{G}_{n+1}$ (mod $\mu$), then
\[
I(X;\mathcal{G}_{n+1}) \ge I(X;\mathcal{G}_n).
\]
If the inclusion is strict, then the inequality is strict.
\end{lemma}

\begin{proof}
Monotonicity follows from the data-processing inequality applied to the
Markov chain
$X \to Y_{n+1} \to Y_n$.
If $\mathcal{G}_n \subsetneq \mathcal{G}_{n+1}$, then
$H(Y_{n+1}\mid Y_n)>0$, implying
$I(X;\mathcal{G}_{n+1})>I(X;\mathcal{G}_n)$.
\end{proof}

\section{Finite Capacity Assumption}

\begin{definition}[Finite Information Capacity]
The refinement process has finite information capacity if there exists
a constant $C<\infty$ such that
\[
\sup_{n} I(X;\mathcal{G}_n) \le C.
\]
\end{definition}

\section{Coercivity of Refinement}

\begin{theorem}[Refinement Coercivity]
Let $\{R_n\}_{n\ge 1}$ be a refinement-driven process with finite
information capacity. Then there exists a constant $K>0$ such that
for all $n$,
\[
I(X;\mathcal{G}_{n+1}) - I(X;\mathcal{G}_n) \le K.
\]
In particular, entropy and mutual information cannot decrease faster
than linearly in the number of admissible refinement steps.
\end{theorem}

\begin{proof}
Each refinement step extracts at most $K$ bits of information by
assumption of finite capacity and bounded observable dimension.
Summing yields
\[
I(X;\mathcal{G}_n) \le I(X;\mathcal{G}_0) + nK \le C,
\]
so the total number of admissible refinements required to achieve a
given entropy reduction grows at least linearly.
\end{proof}

\begin{remark}
This theorem does \emph{not} assert finite termination.
Infinitely many strict refinements may occur, provided each yields
sufficiently small information gain.
\end{remark}

\section{Interpretation for Hamiltonian Dynamics}

In Liouville systems, measure preservation expresses conservation of
phase-space volume. The theorem shows that no refinement scheme with
finite information capacity can extract global structure faster than
linearly.

This establishes a coercive rigidity on adaptive reduced descriptions
of Hamiltonian dynamics.

\section{Conclusion}

We have proved a coercivity principle for refinement-driven descriptions
of Liouville-type systems. Modeling refinement as
$\sigma$-algebra growth clarifies the minimal assumptions required for
the obstruction: bounded information extraction enforces bounded
refinement speed.

The result is structural, model-independent, and mathematically sharp.

\begin{thebibliography}{2}

\bibitem{Shannon1948}
C.~E.~Shannon,
\newblock A mathematical theory of communication,
\newblock \emph{Bell System Technical Journal} \textbf{27} (1948), 379--423,
623--656.

\bibitem{CoverThomas}
T.~M.~Cover and J.~A.~Thomas,
\newblock \emph{Elements of Information Theory},
\newblock Wiley, 2nd edition, 2006.

\end{thebibliography}

\end{document}

